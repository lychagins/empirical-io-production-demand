\subsection{Olley \& Pakes}

%\begin{frame}{A quick recap}
%	We would like to estimate production function (in logs):
%	\be
%		q_{jt} = \beta_0 + \beta_kk_{jt} + \beta_\ell\ell_{jt} + e_{jt}
%	\ee
%	Issues:
%	\bi
%		\item{}Inputs are endogenous; in particular, $\ell_{jt}$ depends on $e_{jt}$
%		\item{}Sample selection: firms make endogenous entry/exit decisions, we only observe continuing firms
%	\ei
%	Estimators
%	\bi
%		\item{OLS: should be avoided -- addresses neither of these issues}
%		\item{Fixed effects: makes strong assumptions about $e_{jt}$; the estimates suggest a substantial bias}
%	\ei
%\end{frame}

\begin{frame}{Olley, Pakes (1996)}
	Overview
	\bi
		\item{A method for the production function estimation allowing for:}
		\bi
			\item{Endogeneity of inputs (labor; capital is pre-determined)}
			\item{Selection (exit)}
			\item{Unobserved productivity differences across firms. Productivity is allowed to change in time with a degree of randomness}
		\ei
		\item{Main requirements of their method:}
		\bi
			\item{The econometrician observes firm level investments into physical capital}
			\item{Investment is a \emph{monotonic} function of a firm's productivity}
			\item{Exit is determined by the firm's productivity}
		\ei
	\ei
\end{frame}


\begin{frame}{Model}
	Firms
	\bi
		\item{Compete in a differentiated goods market}
		\item{Are characterized by \emph{firm-specific state variables}: $(a_{jt}, k_{jt}, u_{jt})$ -- age, capital, efficiency}
		\item{Face same factor prices, which follow an exogenous 1st order Markov process}
	\ei
	There is an infinite number of discrete time periods. Every time period proceeds in three stages:
	\begin{enumerate}
		\item{Each firm learns its new efficiency, $u_{jt}$, and the commonly observed market structure, $\Delta_t$ (factor prices, states of active firms)}
		\item{Firms decide whether to exit. Exiting firms receive $\Phi$ -- a sell-off value -- and never reappear again}
		\item{Continuing firms choose levels of labor and investment}
	\end{enumerate}
	
\end{frame}

\begin{frame}{State transition}
	Tomorrow's firm-specific state depends on the today's state:
	\bi
		\item{$a_{jt+1} = a_{jt} + 1$}
		\item{$K_{jt+1} = (1-\delta)K_{jt} + i_{jt}$ (note the delayed effect)}
		\item{Efficiency index $u_{jt}$ -- an exogenous 1st order Markov process
		\be
			F(u_{jt+1}|I_{jt}) = F(u_{jt+1}|u_{jt})
		\ee
		$I_{jt}$ -- firm $j$'s full information set. $F(u_{jt+1}|u_{jt})$ stochastically increases in $u_{jt}$}
	\ei
\end{frame}

\begin{frame}{Firm's objective}
	Firms maximize expected discounted value of future profits:
	\begin{multline*}
		V(k_{jt}, a_{jt}, u_{jt}, \Delta_t) = \max\{\Phi, \max_{i_{jt}\geq{0}}[\pi(k_{jt}, a_{jt}, u_{jt}, \Delta_t) - c(i_{jt}, \Delta_t)\\
		+ \beta{}E\left[V(k_{jt+1}, a_{jt+1}, u_{jt+1}, \Delta_{t+1})|k_{jt}, a_{jt}, u_{jt}, i_{jt}, \Delta_{jt}\right]]\}
	\end{multline*}
	Equilibrium solution:
	\bi
		\item{Exit rule:}
		\be
			\chi_{jt} = \left\{
			\begin{array}{ll}
				1 & \text{if $u_{jt}\geq{}\underline{u}(k_{jt}, a_{jt}, \Delta_t) = \underline{u}_t(k_{jt}, a_{jt})$}\\
				0 & \text{otherwise}
			\end{array}\right.
		\ee
		\item{Investment rule: $i_{jt} = i(k_{jt}, a_{jt}, u_{jt}, \Delta_t) = i_t(k_{jt}, a_{jt}, u_{jt})$}
	\ei
	\textbf{Important:} under certain conditions, $i_t(\cdot)$ strictly increases in $u_{jt}$, if $i_t>0$
\end{frame}

\begin{frame}{Identification: simple version (no exit)}
Main idea: use investment to back out productivity
\be
	i_{jt} = i_t(k_{jt}, a_{jt}, u_{jt}) \Longrightarrow u_{jt} = h_t(k_{jt}, a_{jt}, i_{jt})
\ee
Note: this inversion cannot be done for observations with $i_{jt} = 0$!\\\

Substitute $u_{jt}$ into the production function:
\begin{multline*}
	q_{jt} = \beta_0 + \beta_kk_{jt} + \beta_\ell\ell_{jt} + \beta_aa_{jt} + u_{jt} + w_{jt}\\
	= \beta_0 + \beta_kk_{jt} + \beta_aa_{jt} + \beta_\ell\ell_{jt} + h_t(k_{jt}, a_{jt}, i_{jt}) + w_{jt}
\end{multline*}
Explanatory variables are independent of $w_{jt}$! $h_t(\cdot)$ may be fairly complex -- treat it as an unknown function.\\\bigskip

Cannot identify $\beta_0$, $\beta_k$, or $\beta_a$: e.g., $(\beta_0, h_t)$ and $(\beta'_0, h_t') = (\beta_0+1, h_t-1)$ will give observationally equivalent models.
\end{frame}

\begin{frame}{Identification: simple version (no exit)}
	Two steps:
	\bi
		\item[Step 1]{Denote $\phi_t(k_{jt}, a_{jt}, i_{jt}) = \beta_0 + \beta_kk_{jt} + \beta_aa_{jt} + h_t(k_{jt}, a_{jt}, i_{jt})$. Estimate}
		\be
			q_{jt} = \beta_\ell\ell_{jt} + \phi_t(k_{jt}, a_{jt}, i_{jt}) + w_{jt}
		\ee
		semi-parametrically, get $\widehat{\beta}_\ell$ and $\widehat{\phi}_t(\cdot)$		
		\item[Step 2]{Split $u_{jt}$: expected part + innovation: 
		\be
			u_{jt} = E[u_{jt}|I_{jt-1}] + \xi_{jt} = E[u_{jt}|u_{jt-1}] + \xi_{jt} = g(u_{jt-1}) + \xi_{jt}
		\ee
		Substitute:
		\begin{align*}
			q_{jt} - \beta_\ell\ell_{jt} &= \beta_0 + \beta_kk_{jt} + \beta_aa_{jt} + g(u_{jt-1}) + \xi_{jt} + w_{jt}\\
			&= \beta_kk_{jt} + \beta_aa_{jt} + \widetilde{g}(\phi_{t-1} - \beta_kk_{jt-1} - \beta_aa_{jt-1}) + \xi_{jt} + w_{jt}
		\end{align*}
		$\widetilde{g}(\cdot)$ is unknown, $\xi_{jt}$ is mean-independent of $(k_{jt}, a_{jt}, i_{jt-1})$ and their lags ($\xi_{jt}$ is a ``surprise'' efficiency).\\
		
		Estimate the last equation semi-parametrically, get $\widehat{\beta}_k$ and $\widehat{\beta}_a$}
	\ei
\end{frame}

\begin{frame}{Identification: extended version (endogenous exit)}
	\bi
	\item[Step 1]{
	No changes in the identification procedure. Exit decisions depend on $i_{jt}$ (via $u_{jt}$)), $k_{jt}$, and $a_{jt}$ --- selection on covariates; doesn't bias the estimates.
	\be
			q_{jt} = \beta_\ell\ell_{jt} + \phi_t(k_{jt}, a_{jt}, i_{jt}) + w_{jt}
	\ee}
	\item[Step 2]{
	Now we have to condition everything on firm's survival:
	\be
		E[q_{jt} - \beta_\ell\ell_{jt}| I_{jt-1}, \chi_{jt}=1] = \beta_0 + \beta_kk_{jt} + \beta_aa_{jt} + E[u_{jt}|I_{jt-1}, \chi_{jt}=1]
	\ee}
	How to find the last term?
	\ei
\end{frame}

\begin{frame}{Identification: extended version, step 2}
	\begin{align*}
		E[u_{jt}|I_{jt-1}, \chi_{jt}=1] &= E[u_{jt}|I_{jt-1}, u_{jt}\geq\underline{u}_t(k_{jt}, a_{jt})]\\
		& = E[u_{jt}|u_{jt-1}, a_{jt}, k_{jt}, u_{jt}\geq\underline{u}_t(k_{jt}, a_{jt})] \\
		& = \int_{\underline{u}_t(k_{jt}, a_{jt})}^\infty\frac{u_{jt}dF(u_{jt}|u_{jt-1})}{\Pr\{u_{jt}\geq\underline{u}_t(k_{jt}, a_{jt})|u_{jt-1},k_{jt}, a_{jt}\}}\\
		&= g(u_{jt-1}, \underline{u}_t(k_{jt}, a_{jt}))
	\end{align*}
	Average productivity conditional on survival depends on the cutoff productivity, below which firms exit. How to find this cutoff?
	\begin{align*}
		P_{jt}= \Pr\{\chi_{jt}=1&|I_{jt-1}\} = \Pr\{u_{jt}\geq\underline{u}_t(k_{jt}, a_{jt})|u_{jt-1}, k_{jt}, a_{jt}\}\\
		& = \Pr\{\chi_{jt}=1|u_{jt-1}, \underline{u}_t(k_{jt}, a_{jt})\} = \widetilde{\widetilde{\varphi}}_t(u_{jt-1}, \underline{u}_t(k_{jt}, a_{jt}))\\
		& = \widetilde{\varphi}_t(u_{jt-1}, k_{jt}, a_{jt}) = \varphi_t(i_{jt-1}, k_{jt-1}, a_{jt-1})
	\end{align*}
	One can identify $\varphi_t(\cdot)$ non-parametrically by computing the percentage of exiting firms for each $(i_{jt-1}, k_{jt-1}, a_{jt-1})$. Then compute $P_{jt}$ for each firm and repeat the inversion trick:
	\begin{align*}
		&\widetilde{\widetilde{\varphi}}_t(u_{jt-1},  \underline{u}_t(k_{jt}, a_{jt})) = P_{jt} \Longrightarrow{}\underline{u}_t(k_{jt}, a_{jt}) = f(u_{jt-1}, P_{jt})
	\end{align*}
\end{frame}

\begin{frame}{Identification: extended version, step 2}
	Substitute everything into the original equation:
	\begin{align*}
		E[q_{jt} &- \beta_\ell\ell_{jt}| I_{jt-1}, \chi_{jt}=1] = \beta_0 + \beta_kk_{jt} + \beta_aa_{jt} + E[u_{jt}|I_{jt-1}, \chi_{jt}=1]\\
		&= \beta_0 + \beta_kk_{jt} + \beta_aa_{jt} + g(u_{jt-1}, \underline{u}_t(k_{jt}, a_{jt}))\\
		&= \beta_0 + \beta_kk_{jt} + \beta_aa_{jt} + g(u_{jt-1}, f(u_{jt-1}, P_{jt}))\\
		&= \beta_kk_{jt} + \beta_aa_{jt} + \widetilde{g}(u_{jt-1}, P_{jt})\\
		&= \beta_kk_{jt} + \beta_aa_{jt} + \widetilde{g}(\phi_{t-1} - \beta_kk_{jt-1} - \beta_aa_{jt-1}, P_{jt})
	\end{align*}
	Define $\eta_{jt}$ as an unexpected shock discovered by the continuing firms:
	\be
		q_{jt} - \beta_\ell\ell_{jt} = \beta_kk_{jt} + \beta_aa_{jt} + \widetilde{g}(\phi_{t-1} - \beta_kk_{jt-1} - \beta_aa_{jt-1}, P_{jt}) + \eta_{jt}
	\ee
	By definition, $\eta_{jt}$ is orthogonal to $I_{jt-1}$ \textbf{conditional on survival} in period $t$ (in particular, to $k_{jt}$, $a_{jt}$ and $P_{jt}$)\\\medskip
	
	Estimate the latter equation semiparametrically, replacing $\phi$ and $P$ with $\widehat{\phi}$ and $\widehat{P}$ $\Longrightarrow$ Find $\widehat{\beta}_k$ and $\widehat{\beta}_a$.
\end{frame}

\begin{frame}{Some notes}
	\bi
		\item{OP estimator requires positive $i_{jt-1}$ -- observations with zero $i_{jt-1}$ are removed from the sample. Does this cause selection bias?}
		\bi
			\item{1st stage equation? No, $i_{jt}$ and the error ($w_{jt}$) are mean independent}
			\item{2nd stage equation? No, $i_{jt-1}\in{I_{jt-1}}$ $\Longrightarrow$ the error ($\eta_{jt}$) is mean independent of $i_{jt-1}$}
			\item{However, this makes us discard valuable data}
		\ei
		\item{Specification test: $\theta = \beta_k/\beta_a$ is identified from two sources:}
		\ei
			\be
		q_{jt} - \beta_\ell\ell_{jt} = \beta_a\left(a_{jt} + \frac{\beta_k}{\beta_a}k_{jt}\right) + \widetilde{g}\left(\phi_{t-1} - \beta_a\left(a_{jt-1} + \frac{\beta_k}{\beta_a}k_{jt-1}\right), P_{jt}\right) + \eta_{jt}
			\ee
Use each source separately to find $\theta_1$ and $\theta_2$:
			\be
		q_{jt} - \beta_\ell\ell_{jt} = \beta_a\left(a_{jt} + \theta_1k_{jt}\right) + \widetilde{g}\left(\phi_{t-1} - \beta_a\left(a_{jt-1} + \theta_2k_{jt-1}\right), P_{jt}\right) + \eta_{jt}
			\ee
and then test whether $\theta_1 = \theta_2$. They should be equal if the model's right.
\end{frame}

\begin{frame}{The setting}
	\bi
		\item{}Telecom equipment industry in 1972--87
		\item{}Major restructuring since late 1960s
		\bi
			\item{Many new products (fax, modem, fiber optics, digital switches, etc.)}
			\item{Deregulation and breakup of AT\&T monopoly}
		\ei
		\item{}De-facto monopoly in the equipment industry until 1980s:
		\bi
			\item{AT\&T procured 90\% of its equipment from own subsidiary, Western Electric}
			\item{AT\&T customers were legally prohibited from connecting own equipment to the network}
		\ei
	\ei
	
\end{frame}

\begin{frame}
	\bi
		\item{}1968 -- Carterphone decision, 1975 -- FCC certification program: customer-owned devices may be connected directly to the network, as long as they do not cause harm to the system
		\item{late 1960s -- 1970s -- a surge of entry into telecom equipment business}
		\item{}1982 -- ``United States v. AT\&T'' settlement: AT\&T to be split up
		\item{}1984 -- Deregulation: 7 regional operating companies (Baby Bells), with no manufacturing capacity (prohibited by the agreement)
		\item{1980s -- Rising competition from foreign entrants}
	\ei
\end{frame}

\begin{frame}{The question}
	Any apparent effects on industry productivity?
	\bi
		\item{}Aggregate productivity increased
	\ei
	What was the mechanism:
	\bi
		\item{}Improvements at the existing plants?
		\item{}Reallocation of labor to more efficient plants?
		\item{}Reallocation of capital to more efficient plants?
	\ei
\end{frame}

\begin{frame}{Estimates}
	\begin{center}
	\begin{tabular}{|l|ccc|}\hline
		Estimator & OLS	& Fixed effects	& Olley-Pakes\\\hline
		Labor 	& 0.693	& 0.629 			& 0.608\\
				& (0.019)	& (0.026) 		& (0.027)\\
		Capital	& 0.304	&  0.150 			& 0.342\\
				& (0.018)	&  (0.026) 		& (0.035)\\
		Age		& -0.0046 & -0.008			& -0.001\\
				& (0.0026)& (0.017)			& (0.004)\\\hline
	\end{tabular}\\
	{\footnotesize{}Source: Olley \& Pakes (1996)}
	\end{center}
	\bi
		\item{The magnitude of $\widehat{\beta}_k$ is more reasonable}
		\item{Compare estimated returns to scale}
	\ei	
\end{frame}

\begin{frame}{Telecom industry productivity}
	Aggregate industry productivity -- share-weighted average of the plant productivity:
	\be
		p_t = \sum_j\exp(q_{jt} - \widehat{\beta}_ll_{jt} - \widehat{\beta}_kk_{jt} - \widehat{\beta}_aa_{jt})\frac{Q_{jt}}{Q_t}
	\ee
	\begin{center}
	\begin{tabular}{|l|c|c|}\hline
		Time period 	& Annual productivity & Comments\\
					& growth rate &\\\hline
		1974--1975	& -0.279	& ???\\
		1975--1977	& 0.020 	&\\
		1978--1980	& 0.146 	&\\
		1981--1983	& -0.087	& AT\&T breakup\\
		1984--1987	& 0.041	&\\\hline
	\end{tabular}
	\end{center}
	Further exercises: growth is mainly caused by reallocation of fixed assets between firms.\\
	
	If you have spare time, check Melitz, Polanec (2015) on growth decomposition formulae. Chance are you will use one of them at some point in your career.
\end{frame}